% Options for packages loaded elsewhere
% Options for packages loaded elsewhere
\PassOptionsToPackage{unicode}{hyperref}
\PassOptionsToPackage{hyphens}{url}
\PassOptionsToPackage{dvipsnames,svgnames,x11names}{xcolor}
%
\documentclass[
]{agujournal2019}
\usepackage{xcolor}
\usepackage{amsmath,amssymb}
\setcounter{secnumdepth}{3}
\usepackage{iftex}
\ifPDFTeX
  \usepackage[T1]{fontenc}
  \usepackage[utf8]{inputenc}
  \usepackage{textcomp} % provide euro and other symbols
\else % if luatex or xetex
  \usepackage{unicode-math} % this also loads fontspec
  \defaultfontfeatures{Scale=MatchLowercase}
  \defaultfontfeatures[\rmfamily]{Ligatures=TeX,Scale=1}
\fi
\usepackage{lmodern}
\ifPDFTeX\else
  % xetex/luatex font selection
\fi
% Use upquote if available, for straight quotes in verbatim environments
\IfFileExists{upquote.sty}{\usepackage{upquote}}{}
\IfFileExists{microtype.sty}{% use microtype if available
  \usepackage[]{microtype}
  \UseMicrotypeSet[protrusion]{basicmath} % disable protrusion for tt fonts
}{}
\makeatletter
\@ifundefined{KOMAClassName}{% if non-KOMA class
  \IfFileExists{parskip.sty}{%
    \usepackage{parskip}
  }{% else
    \setlength{\parindent}{0pt}
    \setlength{\parskip}{6pt plus 2pt minus 1pt}}
}{% if KOMA class
  \KOMAoptions{parskip=half}}
\makeatother
% Make \paragraph and \subparagraph free-standing
\makeatletter
\ifx\paragraph\undefined\else
  \let\oldparagraph\paragraph
  \renewcommand{\paragraph}{
    \@ifstar
      \xxxParagraphStar
      \xxxParagraphNoStar
  }
  \newcommand{\xxxParagraphStar}[1]{\oldparagraph*{#1}\mbox{}}
  \newcommand{\xxxParagraphNoStar}[1]{\oldparagraph{#1}\mbox{}}
\fi
\ifx\subparagraph\undefined\else
  \let\oldsubparagraph\subparagraph
  \renewcommand{\subparagraph}{
    \@ifstar
      \xxxSubParagraphStar
      \xxxSubParagraphNoStar
  }
  \newcommand{\xxxSubParagraphStar}[1]{\oldsubparagraph*{#1}\mbox{}}
  \newcommand{\xxxSubParagraphNoStar}[1]{\oldsubparagraph{#1}\mbox{}}
\fi
\makeatother

\usepackage{color}
\usepackage{fancyvrb}
\newcommand{\VerbBar}{|}
\newcommand{\VERB}{\Verb[commandchars=\\\{\}]}
\DefineVerbatimEnvironment{Highlighting}{Verbatim}{commandchars=\\\{\}}
% Add ',fontsize=\small' for more characters per line
\usepackage{framed}
\definecolor{shadecolor}{RGB}{241,243,245}
\newenvironment{Shaded}{\begin{snugshade}}{\end{snugshade}}
\newcommand{\AlertTok}[1]{\textcolor[rgb]{0.68,0.00,0.00}{#1}}
\newcommand{\AnnotationTok}[1]{\textcolor[rgb]{0.37,0.37,0.37}{#1}}
\newcommand{\AttributeTok}[1]{\textcolor[rgb]{0.40,0.45,0.13}{#1}}
\newcommand{\BaseNTok}[1]{\textcolor[rgb]{0.68,0.00,0.00}{#1}}
\newcommand{\BuiltInTok}[1]{\textcolor[rgb]{0.00,0.23,0.31}{#1}}
\newcommand{\CharTok}[1]{\textcolor[rgb]{0.13,0.47,0.30}{#1}}
\newcommand{\CommentTok}[1]{\textcolor[rgb]{0.37,0.37,0.37}{#1}}
\newcommand{\CommentVarTok}[1]{\textcolor[rgb]{0.37,0.37,0.37}{\textit{#1}}}
\newcommand{\ConstantTok}[1]{\textcolor[rgb]{0.56,0.35,0.01}{#1}}
\newcommand{\ControlFlowTok}[1]{\textcolor[rgb]{0.00,0.23,0.31}{\textbf{#1}}}
\newcommand{\DataTypeTok}[1]{\textcolor[rgb]{0.68,0.00,0.00}{#1}}
\newcommand{\DecValTok}[1]{\textcolor[rgb]{0.68,0.00,0.00}{#1}}
\newcommand{\DocumentationTok}[1]{\textcolor[rgb]{0.37,0.37,0.37}{\textit{#1}}}
\newcommand{\ErrorTok}[1]{\textcolor[rgb]{0.68,0.00,0.00}{#1}}
\newcommand{\ExtensionTok}[1]{\textcolor[rgb]{0.00,0.23,0.31}{#1}}
\newcommand{\FloatTok}[1]{\textcolor[rgb]{0.68,0.00,0.00}{#1}}
\newcommand{\FunctionTok}[1]{\textcolor[rgb]{0.28,0.35,0.67}{#1}}
\newcommand{\ImportTok}[1]{\textcolor[rgb]{0.00,0.46,0.62}{#1}}
\newcommand{\InformationTok}[1]{\textcolor[rgb]{0.37,0.37,0.37}{#1}}
\newcommand{\KeywordTok}[1]{\textcolor[rgb]{0.00,0.23,0.31}{\textbf{#1}}}
\newcommand{\NormalTok}[1]{\textcolor[rgb]{0.00,0.23,0.31}{#1}}
\newcommand{\OperatorTok}[1]{\textcolor[rgb]{0.37,0.37,0.37}{#1}}
\newcommand{\OtherTok}[1]{\textcolor[rgb]{0.00,0.23,0.31}{#1}}
\newcommand{\PreprocessorTok}[1]{\textcolor[rgb]{0.68,0.00,0.00}{#1}}
\newcommand{\RegionMarkerTok}[1]{\textcolor[rgb]{0.00,0.23,0.31}{#1}}
\newcommand{\SpecialCharTok}[1]{\textcolor[rgb]{0.37,0.37,0.37}{#1}}
\newcommand{\SpecialStringTok}[1]{\textcolor[rgb]{0.13,0.47,0.30}{#1}}
\newcommand{\StringTok}[1]{\textcolor[rgb]{0.13,0.47,0.30}{#1}}
\newcommand{\VariableTok}[1]{\textcolor[rgb]{0.07,0.07,0.07}{#1}}
\newcommand{\VerbatimStringTok}[1]{\textcolor[rgb]{0.13,0.47,0.30}{#1}}
\newcommand{\WarningTok}[1]{\textcolor[rgb]{0.37,0.37,0.37}{\textit{#1}}}

\usepackage{longtable,booktabs,array}
\usepackage{calc} % for calculating minipage widths
% Correct order of tables after \paragraph or \subparagraph
\usepackage{etoolbox}
\makeatletter
\patchcmd\longtable{\par}{\if@noskipsec\mbox{}\fi\par}{}{}
\makeatother
% Allow footnotes in longtable head/foot
\IfFileExists{footnotehyper.sty}{\usepackage{footnotehyper}}{\usepackage{footnote}}
\makesavenoteenv{longtable}
\usepackage{graphicx}
\makeatletter
\newsavebox\pandoc@box
\newcommand*\pandocbounded[1]{% scales image to fit in text height/width
  \sbox\pandoc@box{#1}%
  \Gscale@div\@tempa{\textheight}{\dimexpr\ht\pandoc@box+\dp\pandoc@box\relax}%
  \Gscale@div\@tempb{\linewidth}{\wd\pandoc@box}%
  \ifdim\@tempb\p@<\@tempa\p@\let\@tempa\@tempb\fi% select the smaller of both
  \ifdim\@tempa\p@<\p@\scalebox{\@tempa}{\usebox\pandoc@box}%
  \else\usebox{\pandoc@box}%
  \fi%
}
% Set default figure placement to htbp
\def\fps@figure{htbp}
\makeatother


% definitions for citeproc citations
\NewDocumentCommand\citeproctext{}{}
\NewDocumentCommand\citeproc{mm}{%
  \begingroup\def\citeproctext{#2}\cite{#1}\endgroup}
\makeatletter
 % allow citations to break across lines
 \let\@cite@ofmt\@firstofone
 % avoid brackets around text for \cite:
 \def\@biblabel#1{}
 \def\@cite#1#2{{#1\if@tempswa , #2\fi}}
\makeatother
\newlength{\cslhangindent}
\setlength{\cslhangindent}{1.5em}
\newlength{\csllabelwidth}
\setlength{\csllabelwidth}{3em}
\newenvironment{CSLReferences}[2] % #1 hanging-indent, #2 entry-spacing
 {\begin{list}{}{%
  \setlength{\itemindent}{0pt}
  \setlength{\leftmargin}{0pt}
  \setlength{\parsep}{0pt}
  % turn on hanging indent if param 1 is 1
  \ifodd #1
   \setlength{\leftmargin}{\cslhangindent}
   \setlength{\itemindent}{-1\cslhangindent}
  \fi
  % set entry spacing
  \setlength{\itemsep}{#2\baselineskip}}}
 {\end{list}}
\usepackage{calc}
\newcommand{\CSLBlock}[1]{\hfill\break\parbox[t]{\linewidth}{\strut\ignorespaces#1\strut}}
\newcommand{\CSLLeftMargin}[1]{\parbox[t]{\csllabelwidth}{\strut#1\strut}}
\newcommand{\CSLRightInline}[1]{\parbox[t]{\linewidth - \csllabelwidth}{\strut#1\strut}}
\newcommand{\CSLIndent}[1]{\hspace{\cslhangindent}#1}



\setlength{\emergencystretch}{3em} % prevent overfull lines

\providecommand{\tightlist}{%
  \setlength{\itemsep}{0pt}\setlength{\parskip}{0pt}}



 


\usepackage{url} %this package should fix any errors with URLs in refs.
\usepackage{lineno}
\usepackage[inline]{trackchanges} %for better track changes. finalnew option will compile document with changes incorporated.
\usepackage{soul}
\linenumbers
\makeatletter
\@ifpackageloaded{tcolorbox}{}{\usepackage[skins,breakable]{tcolorbox}}
\@ifpackageloaded{fontawesome5}{}{\usepackage{fontawesome5}}
\definecolor{quarto-callout-color}{HTML}{909090}
\definecolor{quarto-callout-note-color}{HTML}{0758E5}
\definecolor{quarto-callout-important-color}{HTML}{CC1914}
\definecolor{quarto-callout-warning-color}{HTML}{EB9113}
\definecolor{quarto-callout-tip-color}{HTML}{00A047}
\definecolor{quarto-callout-caution-color}{HTML}{FC5300}
\definecolor{quarto-callout-color-frame}{HTML}{acacac}
\definecolor{quarto-callout-note-color-frame}{HTML}{4582ec}
\definecolor{quarto-callout-important-color-frame}{HTML}{d9534f}
\definecolor{quarto-callout-warning-color-frame}{HTML}{f0ad4e}
\definecolor{quarto-callout-tip-color-frame}{HTML}{02b875}
\definecolor{quarto-callout-caution-color-frame}{HTML}{fd7e14}
\makeatother
\makeatletter
\@ifpackageloaded{caption}{}{\usepackage{caption}}
\AtBeginDocument{%
\ifdefined\contentsname
  \renewcommand*\contentsname{Table of contents}
\else
  \newcommand\contentsname{Table of contents}
\fi
\ifdefined\listfigurename
  \renewcommand*\listfigurename{List of Figures}
\else
  \newcommand\listfigurename{List of Figures}
\fi
\ifdefined\listtablename
  \renewcommand*\listtablename{List of Tables}
\else
  \newcommand\listtablename{List of Tables}
\fi
\ifdefined\figurename
  \renewcommand*\figurename{Figure}
\else
  \newcommand\figurename{Figure}
\fi
\ifdefined\tablename
  \renewcommand*\tablename{Table}
\else
  \newcommand\tablename{Table}
\fi
}
\@ifpackageloaded{float}{}{\usepackage{float}}
\floatstyle{ruled}
\@ifundefined{c@chapter}{\newfloat{codelisting}{h}{lop}}{\newfloat{codelisting}{h}{lop}[chapter]}
\floatname{codelisting}{Listing}
\newcommand*\listoflistings{\listof{codelisting}{List of Listings}}
\makeatother
\makeatletter
\makeatother
\makeatletter
\@ifpackageloaded{caption}{}{\usepackage{caption}}
\@ifpackageloaded{subcaption}{}{\usepackage{subcaption}}
\makeatother
\usepackage{bookmark}
\IfFileExists{xurl.sty}{\usepackage{xurl}}{} % add URL line breaks if available
\urlstyle{same}
\hypersetup{
  pdftitle={Gloucester, Barton cemetery - radiocarbon dating and chronological modelling},
  pdfauthor={Peter Marshall},
  pdfkeywords={Gloucester, Barton cemetery, Radiocarbon
dating, Chronological modelling},
  colorlinks=true,
  linkcolor={blue},
  filecolor={Maroon},
  citecolor={Blue},
  urlcolor={Blue},
  pdfcreator={LaTeX via pandoc}}



\draftfalse

\begin{document}
\title{Gloucester, Barton cemetery - radiocarbon dating and
chronological modelling}

\authors{Peter Marshall\affil{1}}
\affiliation{1}{Chronologies, }
\correspondingauthor{Peter Marshall}{pete.chronologies@googlemail.com}


\begin{abstract}
Chronological modelling of the available radiocarbon \& coin dates from
the Gloucester, Barton cemetery was undertaken following preliminary
analysis of the dated inhumations diets indicated they all had a
terrestiral diet. A series of simulation models were then run, to
explore the effect of radiocarbon dating individuals from Block M that
had been selcted for isotopic \& aDNA analysis. The results of the
simulation indicate that the submisison of further samples for
radiocarbon dating is not going to improve the precision of estimates
for the chronology of the Block M part of the cemetery.
\end{abstract}




\renewcommand*\contentsname{Table of contents}
{
\hypersetup{linkcolor=}
\setcounter{tocdepth}{3}
\tableofcontents
}

\section{Introduction}\label{introduction}

Between June 2013 and December 2018 Cotswold Archaeology (CA) carried
out an archaeological investigation of the former Gloscat Media Studies
site, Brunswick Road, Gloucester. The fieldwork on the Media Studies
site occurred in two phases. The first phase of excavation covered
0.18ha and took place between June 2013 and November 2014. An additional
0.6ha in the northern corner of the development site (known as Block M)
was investigated between January 2017 and December 2018 (the two
excavation areas were contiguous).

\subsection{Radiocarbon dating}\label{radiocarbon-dating}

Fourteen radiocarbon measurements are available from the Gloscat Media
Studies site (n=10) and Block M (n=4). The Gloscat Media Studies samples
were analysed during March and May 2015 at Scottish Universities
Environmental Research Centre (SUERC; Table~\ref{tbl-01}). The Block M
samples were analysed between November 2020 and February 2021 at the
Bristol Radiocarbon Accelerator Mass Spectrometry (BRAMS) Facility.

Details of the dated samples, radiocarbon ages, and associated stable
isotopic measurements are provided in Table~\ref{tbl-01}. The
radiocarbon results are conventional radiocarbon ages (Stuiver \&
Polach, 1977). All SUERC they were corrected for fractionation using
δ\textsuperscript{13}C values measured during the dating process by
Accelerator Mass Spectrometry (AMS), and at BRAMS using the
δ\textsuperscript{13}C value measured by Isotope Ratio Mass Spectrometry
(IRMS). In addition, δ\textsuperscript{13}C values and
δ\textsuperscript{15}N values have been obtained on sub-samples of the
dated material by Isotope Ratio Mass Spectrometry, as these results more
accurately reflect the natural isotopic composition of the sampled
material.

The pretreatment, combustion, graphitisation, and measurement by AMS of
the sample dated at the Scottish Universities Environmental Research
Centre (SUERC-) followed the methods outlined in (Dunbar et al., 2016)
while those processed and dated by AMS at BRAMS followed the methods
described in (Knowles et al., 2019) and (Synal et al., 2007)

\begin{longtable}[]{@{}
  >{\raggedright\arraybackslash}p{(\linewidth - 10\tabcolsep) * \real{0.1667}}
  >{\raggedright\arraybackslash}p{(\linewidth - 10\tabcolsep) * \real{0.1667}}
  >{\raggedright\arraybackslash}p{(\linewidth - 10\tabcolsep) * \real{0.1667}}
  >{\raggedright\arraybackslash}p{(\linewidth - 10\tabcolsep) * \real{0.1667}}
  >{\raggedright\arraybackslash}p{(\linewidth - 10\tabcolsep) * \real{0.1667}}
  >{\raggedright\arraybackslash}p{(\linewidth - 10\tabcolsep) * \real{0.1667}}@{}}
\caption{Gloucester Barton cemetery: radiocarbon \& stable isotope
measurements}\label{tbl-01}\tabularnewline
\toprule\noalign{}
\begin{minipage}[b]{\linewidth}\raggedright
Laboratory Code
\end{minipage} & \begin{minipage}[b]{\linewidth}\raggedright
Material \& context
\end{minipage} & \begin{minipage}[b]{\linewidth}\raggedright
Radiocarbon age (BP)
\end{minipage} & \begin{minipage}[b]{\linewidth}\raggedright
\textbf{δ\textsuperscript{13}C\textsubscript{IRMS} (‰)}
\end{minipage} & \begin{minipage}[b]{\linewidth}\raggedright
\textbf{δ\textsuperscript{15}N\textsubscript{IRMS} (‰)}
\end{minipage} & \begin{minipage}[b]{\linewidth}\raggedright
C:N
\end{minipage} \\
\midrule\noalign{}
\endfirsthead
\toprule\noalign{}
\begin{minipage}[b]{\linewidth}\raggedright
Laboratory Code
\end{minipage} & \begin{minipage}[b]{\linewidth}\raggedright
Material \& context
\end{minipage} & \begin{minipage}[b]{\linewidth}\raggedright
Radiocarbon age (BP)
\end{minipage} & \begin{minipage}[b]{\linewidth}\raggedright
\textbf{δ\textsuperscript{13}C\textsubscript{IRMS} (‰)}
\end{minipage} & \begin{minipage}[b]{\linewidth}\raggedright
\textbf{δ\textsuperscript{15}N\textsubscript{IRMS} (‰)}
\end{minipage} & \begin{minipage}[b]{\linewidth}\raggedright
C:N
\end{minipage} \\
\midrule\noalign{}
\endhead
\bottomrule\noalign{}
\endlastfoot
SUERC-58710 & Human~bone:~right~arm.~Sk7494.~Well~7481 & 1649±31 &
-20.1±0.2 & 11.1±0.3 & 3.3 \\
SUERC-58711 & Human~bone:~right~arm.~Sk6191.~Grave~6143 & 1748±29 &
-19.5±0.2 & 11.4±0.3 & 3.3 \\
SUERC-58712 & Human~bone:~left~arm.~Sk6120.~Grave~6119 & 1855±31 &
-20.7±0.2 & 10.6±0.3 & 3.5 \\
SUERC-58713 & Human~bone:~right~arm.~Sk6338.~Grave~6339 & 1759±29 &
-19.7±0.2 & 7.5±0.3 & 3.3 \\
SUERC-58717 & Charcoal:~\emph{Corylus~avellana}~roundwood.~Deposit~8294
& 1874±28 & -24.6±0.2 & NA & NA \\
SUERC-59634 & Human~bone:~left~femur.~Sk6920.~Grave~6918 & 1664±29 &
-19.8±0.2 & 10.8±0.3 & 3.3 \\
SUERC-59806 & Human~bone~(cremated):~Cremation~6445.~Context~6446 &
1713±27 & -25.7±0.2 & NA & NA \\
SUERC-61220 &
Charcoal:~\emph{Alnus~glutinosa/Corylus~avellana}.~Pit~6108.~Context~6109
& 1920±29 & -23.6±0.2 & NA & NA \\
SUERC-61221 &
Charcoal:~\emph{Alnus~glutinosa/Corylus~avellana}.~Deposit~6569 &
1927±29 & -24.8±0.2 & NA & NA \\
SUERC-61222 & Charcoal:~\emph{Prunus}.~Ditch~6566.~Context~6565. &
1100±27 & -24.8±0.2 & NA & NA \\
BRAMS-4172 & Human~bone:~rib.~Sk2017.~Grave~2016 & 1751±25 & -19.95±0.2
& 9.87±0.3 & NA \\
BRAMS-4173 & Human~bone:~rib.~Sk1819.~Grave~1820 & 1778±25 & -19.69±0.2
& 10.39±0.3 & NA \\
BRAMS-4174 & Human~bone:~right~fibula.~Sk1897.~Grave~1898 & 1739±25 &
-19.94±0.2 & 10.14±0.3 & NA \\
BRAMS-4175 & Human~bone:~right~rib.~Sk1873.~Grave~1874 & 1967±27 &
-19.72±0.2 & 9.38±0.3 & NA \\
\end{longtable}

Table~\ref{tbl-01} Gloucester Barton cemetery: radiocarbon \& stable
isotope measurements

\subsection{Assessment of existing radiocarbon
dates}\label{assessment-of-existing-radiocarbon-dates}

\subsubsection{Diet-induced offsets}\label{diet-induced-offsets}

Diet-induced offsets may arise when an animal consumes carbon from a
reservoir that is not in equilibrium with the terrestrial biosphere
Lanting \& Plicht (1998). Should one of the dietary reservoir sources
possess an intrinsic radiocarbon offset, the animal will incorporate a
proportion of radiocarbon that deviates from atmospheric equilibrium.
Radiocarbon ages derived from such sources, if incorrectly calibrated
using a solely terrestrial calibration curve, will yield anomalously old
age estimates.

Figure~\ref{fig-01} and Figure~\ref{fig-02} summarise the
δ\textsuperscript{13}C and δ\textsuperscript{15}N values from human bone
radiocarbon samples. These indicate that the dated individuals consumed
a terrestrial based diet and the radiocarbon results are unlikely to be
affected by any significant reservoir effects, so a fully terrestrial
calibration curve can be employed.

\begin{figure}

\centering{

\pandocbounded{\includegraphics[keepaspectratio]{images/isotope_bivariate_plot.png}}

}

\caption{\label{fig-01}Bivariate plot of \emph{δ}\textsuperscript{13}C
and \emph{δ}\textsuperscript{15}N stable isotopic values of bone
collagen}

\end{figure}%

\begin{Shaded}
\begin{Highlighting}[]
\CommentTok{\# Load required libraries}
\FunctionTok{library}\NormalTok{(readxl)}
\FunctionTok{library}\NormalTok{(dplyr)}
\FunctionTok{library}\NormalTok{(ggplot2)}

\CommentTok{\# Read the Excel file}
\CommentTok{\# Change *.xlsx to the name of your Excel file}
\CommentTok{\# Change sheet= to the name of the sheet with the data on it you want to plot}
\NormalTok{gloscat }\OtherTok{\textless{}{-}} \FunctionTok{read\_excel}\NormalTok{(}\StringTok{"Gloscat.xlsx"}\NormalTok{, }\AttributeTok{sheet =} \StringTok{"Data"}\NormalTok{) }\CommentTok{\# change data to name of site name or something else}

\CommentTok{\# Filter for human bone samples with complete isotope data (use spreadsheet template otherwise change column headings)}
\NormalTok{human\_bone\_complete }\OtherTok{\textless{}{-}}\NormalTok{ gloscat }\SpecialCharTok{\%\textgreater{}\%} \CommentTok{\# change human\_bone\_complete to site name or something else}
  \FunctionTok{filter}\NormalTok{(Material }\SpecialCharTok{==} \StringTok{"Human bone"}\NormalTok{,}
         \SpecialCharTok{!}\FunctionTok{is.na}\NormalTok{(d13C), }\SpecialCharTok{!}\FunctionTok{is.na}\NormalTok{(d13Cerror),}
         \SpecialCharTok{!}\FunctionTok{is.na}\NormalTok{(d15N), }\SpecialCharTok{!}\FunctionTok{is.na}\NormalTok{(d15Nerror)) }\SpecialCharTok{\%\textgreater{}\%}
  \FunctionTok{select}\NormalTok{(LabCode, d13C, d13Cerror, d15N, d15Nerror)}

\CommentTok{\# View the filtered data}
\FunctionTok{print}\NormalTok{(human\_bone\_complete) }\CommentTok{\# if you have changed human\_bone\_complete change here}
\FunctionTok{print}\NormalTok{(}\FunctionTok{paste}\NormalTok{(}\StringTok{"Number of complete samples:"}\NormalTok{, }\FunctionTok{nrow}\NormalTok{(human\_bone\_complete))) }

\CommentTok{\# Create δ13C vs δ15N bivariate plot with error bars}
\NormalTok{isotope\_plot }\OtherTok{\textless{}{-}} \FunctionTok{ggplot}\NormalTok{(human\_bone\_complete, }\FunctionTok{aes}\NormalTok{(}\AttributeTok{x =}\NormalTok{ d13C, }\AttributeTok{y =}\NormalTok{ d15N)) }\SpecialCharTok{+} \CommentTok{\# if you have changed human\_bone\_complete change here}
  \CommentTok{\# Add error bars}
  \FunctionTok{geom\_errorbar}\NormalTok{(}\FunctionTok{aes}\NormalTok{(}\AttributeTok{ymin =}\NormalTok{ d15N }\SpecialCharTok{{-}}\NormalTok{ d15Nerror, }\AttributeTok{ymax =}\NormalTok{ d15N }\SpecialCharTok{+}\NormalTok{ d15Nerror), }
                \AttributeTok{width =} \FloatTok{0.1}\NormalTok{, }\AttributeTok{color =} \StringTok{"gray50"}\NormalTok{, }\AttributeTok{alpha =} \FloatTok{0.7}\NormalTok{) }\SpecialCharTok{+}
  \FunctionTok{geom\_errorbarh}\NormalTok{(}\FunctionTok{aes}\NormalTok{(}\AttributeTok{xmin =}\NormalTok{ d13C }\SpecialCharTok{{-}}\NormalTok{ d13Cerror, }\AttributeTok{xmax =}\NormalTok{ d13C }\SpecialCharTok{+}\NormalTok{ d13Cerror), }
                 \AttributeTok{height =} \FloatTok{0.1}\NormalTok{, }\AttributeTok{color =} \StringTok{"gray50"}\NormalTok{, }\AttributeTok{alpha =} \FloatTok{0.7}\NormalTok{) }\SpecialCharTok{+}
  \CommentTok{\# Add data points}
  \FunctionTok{geom\_point}\NormalTok{(}\AttributeTok{size =} \DecValTok{3}\NormalTok{, }\AttributeTok{color =} \StringTok{"darkblue"}\NormalTok{, }\AttributeTok{alpha =} \FloatTok{0.8}\NormalTok{) }\SpecialCharTok{+}
  \CommentTok{\# Add labels for each point}
  \FunctionTok{geom\_text}\NormalTok{(}\FunctionTok{aes}\NormalTok{(}\AttributeTok{label =}\NormalTok{ LabCode), }
            \AttributeTok{vjust =} \SpecialCharTok{{-}}\FloatTok{0.5}\NormalTok{, }\AttributeTok{hjust =} \FloatTok{0.5}\NormalTok{, }\AttributeTok{size =} \DecValTok{3}\NormalTok{, }\AttributeTok{color =} \StringTok{"black"}\NormalTok{) }\SpecialCharTok{+}
  \CommentTok{\# Customize the plot}
  \FunctionTok{labs}\NormalTok{(}\AttributeTok{title =} \StringTok{"Gloucester: Barton cemetery"}\NormalTok{,}
       \AttributeTok{subtitle =} \FunctionTok{expression}\NormalTok{(}\FunctionTok{paste}\NormalTok{(}\StringTok{"Human bone collagen (radiocarbon)"}\NormalTok{)),}
       \AttributeTok{x =} \FunctionTok{expression}\NormalTok{(}\FunctionTok{paste}\NormalTok{(delta}\SpecialCharTok{\^{}}\DecValTok{13}\NormalTok{, }\StringTok{"C (per mille)"}\NormalTok{)),}
       \AttributeTok{y =} \FunctionTok{expression}\NormalTok{(}\FunctionTok{paste}\NormalTok{(delta}\SpecialCharTok{\^{}}\DecValTok{15}\NormalTok{, }\StringTok{"N (per mille)"}\NormalTok{))) }\SpecialCharTok{+}
  \FunctionTok{theme\_classic}\NormalTok{() }\SpecialCharTok{+}
  \FunctionTok{theme}\NormalTok{(}\AttributeTok{plot.title =} \FunctionTok{element\_text}\NormalTok{(}\AttributeTok{size =} \DecValTok{14}\NormalTok{, }\AttributeTok{hjust =} \FloatTok{0.5}\NormalTok{),}
        \AttributeTok{plot.subtitle =} \FunctionTok{element\_text}\NormalTok{(}\AttributeTok{size =} \DecValTok{12}\NormalTok{, }\AttributeTok{hjust =} \FloatTok{0.5}\NormalTok{),}
        \AttributeTok{axis.title =} \FunctionTok{element\_text}\NormalTok{(}\AttributeTok{size =} \DecValTok{12}\NormalTok{),}
        \AttributeTok{axis.text =} \FunctionTok{element\_text}\NormalTok{(}\AttributeTok{size =} \DecValTok{10}\NormalTok{),}
        \AttributeTok{panel.grid.minor =} \FunctionTok{element\_blank}\NormalTok{())}

\CommentTok{\# Display the plot}
\FunctionTok{print}\NormalTok{(isotope\_plot)}

\CommentTok{\# Save the plot}
\FunctionTok{ggsave}\NormalTok{(}\StringTok{"isotope\_bivariate\_plot.png"}\NormalTok{, isotope\_plot, }
       \AttributeTok{width =} \DecValTok{8}\NormalTok{, }\AttributeTok{height =} \DecValTok{6}\NormalTok{, }\AttributeTok{dpi =} \DecValTok{300}\NormalTok{)}
\end{Highlighting}
\end{Shaded}

\begin{figure}

\centering{

\pandocbounded{\includegraphics[keepaspectratio]{images/isotopeplot2.png}}

}

\caption{\label{fig-02}Bivariate plot of \emph{δ}\textsuperscript{13}C
and \emph{δ}\textsuperscript{15}N stable isotopic values of bone
collagen, together with potential sources}

\end{figure}%

\begin{Shaded}
\begin{Highlighting}[]
\CommentTok{\#Add Food source ellipses \& remove individual data labels}

\CommentTok{\# Read the Excel file for food source data}
\NormalTok{gloscat\_sources }\OtherTok{\textless{}{-}} \FunctionTok{read\_excel}\NormalTok{(}\StringTok{"Gloscat.xlsx"}\NormalTok{, }\AttributeTok{sheet =} \StringTok{"Baseline"}\NormalTok{) }\CommentTok{\# change data to name of site name or something else}

\CommentTok{\# create plot without individual data labels}
\NormalTok{isotope\_plot2 }\OtherTok{\textless{}{-}} \FunctionTok{ggplot}\NormalTok{(human\_bone\_complete, }\FunctionTok{aes}\NormalTok{(}\AttributeTok{x =}\NormalTok{ d13C, }\AttributeTok{y =}\NormalTok{ d15N)) }\SpecialCharTok{+} \CommentTok{\# if you have changed human\_bone\_complete change here}
  \CommentTok{\# Add error bars}
  \FunctionTok{geom\_errorbar}\NormalTok{(}\FunctionTok{aes}\NormalTok{(}\AttributeTok{ymin =}\NormalTok{ d15N }\SpecialCharTok{{-}}\NormalTok{ d15Nerror, }\AttributeTok{ymax =}\NormalTok{ d15N }\SpecialCharTok{+}\NormalTok{ d15Nerror), }
                \AttributeTok{width =} \FloatTok{0.1}\NormalTok{, }\AttributeTok{color =} \StringTok{"gray50"}\NormalTok{, }\AttributeTok{alpha =} \FloatTok{0.7}\NormalTok{) }\SpecialCharTok{+}
  \FunctionTok{geom\_errorbarh}\NormalTok{(}\FunctionTok{aes}\NormalTok{(}\AttributeTok{xmin =}\NormalTok{ d13C }\SpecialCharTok{{-}}\NormalTok{ d13Cerror, }\AttributeTok{xmax =}\NormalTok{ d13C }\SpecialCharTok{+}\NormalTok{ d13Cerror), }
                 \AttributeTok{height =} \FloatTok{0.1}\NormalTok{, }\AttributeTok{color =} \StringTok{"gray50"}\NormalTok{, }\AttributeTok{alpha =} \FloatTok{0.7}\NormalTok{) }\SpecialCharTok{+}
  \CommentTok{\# Add data points}
  \FunctionTok{geom\_point}\NormalTok{(}\AttributeTok{size =} \DecValTok{3}\NormalTok{, }\AttributeTok{color =} \StringTok{"darkblue"}\NormalTok{, }\AttributeTok{alpha =} \FloatTok{0.8}\NormalTok{) }\SpecialCharTok{+}
  \CommentTok{\# Add labels for each point}
  \CommentTok{\# Customize the plot}
  \FunctionTok{labs}\NormalTok{(}\AttributeTok{title =} \StringTok{"Gloucester: Barton cemtery"}\NormalTok{,}
       \AttributeTok{subtitle =} \FunctionTok{expression}\NormalTok{(}\FunctionTok{paste}\NormalTok{(}\StringTok{"Human bone collagen (radiocarbon)"}\NormalTok{)),}
       \AttributeTok{x =} \FunctionTok{expression}\NormalTok{(}\FunctionTok{paste}\NormalTok{(delta}\SpecialCharTok{\^{}}\DecValTok{13}\NormalTok{, }\StringTok{"C (per mille)"}\NormalTok{)),}
       \AttributeTok{y =} \FunctionTok{expression}\NormalTok{(}\FunctionTok{paste}\NormalTok{(delta}\SpecialCharTok{\^{}}\DecValTok{15}\NormalTok{, }\StringTok{"N (per mille)"}\NormalTok{))) }\SpecialCharTok{+}
  \FunctionTok{theme\_classic}\NormalTok{() }\SpecialCharTok{+}
  \FunctionTok{theme}\NormalTok{(}\AttributeTok{plot.title =} \FunctionTok{element\_text}\NormalTok{(}\AttributeTok{size =} \DecValTok{14}\NormalTok{, }\AttributeTok{hjust =} \FloatTok{0.5}\NormalTok{),}
        \AttributeTok{plot.subtitle =} \FunctionTok{element\_text}\NormalTok{(}\AttributeTok{size =} \DecValTok{12}\NormalTok{, }\AttributeTok{hjust =} \FloatTok{0.5}\NormalTok{),}
        \AttributeTok{axis.title =} \FunctionTok{element\_text}\NormalTok{(}\AttributeTok{size =} \DecValTok{12}\NormalTok{),}
        \AttributeTok{axis.text =} \FunctionTok{element\_text}\NormalTok{(}\AttributeTok{size =} \DecValTok{10}\NormalTok{),}
        \AttributeTok{panel.grid.minor =} \FunctionTok{element\_blank}\NormalTok{())}

\CommentTok{\# add food source ellipses}
\NormalTok{isotope\_plot2}\OtherTok{\textless{}{-}}\NormalTok{ isotope\_plot2 }\SpecialCharTok{+} \FunctionTok{stat\_ellipse}\NormalTok{(}\AttributeTok{data =}\NormalTok{ gloscat\_sources,}
                 \FunctionTok{aes}\NormalTok{(}\AttributeTok{x =}\NormalTok{ d13C, }\AttributeTok{y =}\NormalTok{ d15N, }\AttributeTok{color =}\NormalTok{ Source, }\AttributeTok{fill =}\NormalTok{ Source),}
                 \AttributeTok{type =} \StringTok{"norm"}\NormalTok{, }\AttributeTok{level =} \FloatTok{0.68}\NormalTok{,  }\CommentTok{\# standard ellipse ≈ 1 SD}
                 \AttributeTok{geom =} \StringTok{"polygon"}\NormalTok{, }\AttributeTok{alpha =} \FloatTok{0.2}\NormalTok{, }\AttributeTok{show.legend =} \ConstantTok{TRUE}\NormalTok{)}

\CommentTok{\# Display the plot}
\FunctionTok{print}\NormalTok{(isotope\_plot2)}

\CommentTok{\# Save the plot}
\FunctionTok{ggsave}\NormalTok{(}\StringTok{"isotopeplot2.png"}\NormalTok{, isotope\_plot2, }
       \AttributeTok{width =} \DecValTok{8}\NormalTok{, }\AttributeTok{height =} \DecValTok{6}\NormalTok{, }\AttributeTok{dpi =} \DecValTok{300}\NormalTok{)}

\CommentTok{\#if you want to see all the food source data use this}
\NormalTok{isotope\_plot3}\OtherTok{\textless{}{-}}\NormalTok{isotope\_plot2 }\SpecialCharTok{+} \FunctionTok{geom\_point}\NormalTok{(}\AttributeTok{data =}\NormalTok{ gloscat\_sources,}
                           \FunctionTok{aes}\NormalTok{(}\AttributeTok{x =}\NormalTok{ d13C, }\AttributeTok{y =}\NormalTok{ d15N, }\AttributeTok{color =}\NormalTok{ Source),}
                           \AttributeTok{shape =} \DecValTok{21}\NormalTok{, }\AttributeTok{size =} \DecValTok{3}\NormalTok{, }\AttributeTok{fill =} \StringTok{"white"}\NormalTok{)}

\CommentTok{\# Display the plot}
\FunctionTok{print}\NormalTok{(isotope\_plot3)}

\CommentTok{\# Save the plot}
\FunctionTok{ggsave}\NormalTok{(}\StringTok{"isotopeplot3.pnd"}\NormalTok{, isotope\_plot3, }
       \AttributeTok{width =} \DecValTok{8}\NormalTok{, }\AttributeTok{height =} \DecValTok{6}\NormalTok{, }\AttributeTok{dpi =} \DecValTok{300}\NormalTok{)}
\end{Highlighting}
\end{Shaded}

\subsection{Chronological modelling}\label{chronological-modelling}

The chronological modelling presented here has been undertaken using
OxCal 4.4 (Bronk Ramsey, 2009), and the internationally agreed
calibration curve for the northern hemisphere (IntCal20; Reimer et al.
(2020)). The models are defined by the OxCal keywords and brackets on
the left-hand side of Figure~\ref{fig-03} \& Figure~\ref{fig-04} and by
the CQL2 code provided below. In the figures, calibrated radiocarbon
dates are shown in outline, and the posterior density estimates produced
by the chronological modelling are shown in solid black. The other
distributions correspond to aspects of the model. For example, the
distribution \emph{FirstBurialBlockM} (Figure~\ref{fig-04}) is the
posterior density estimate for when the first dated individual in block
M was interred. In the text and tables highest posterior density
intervals, which describe the posterior distributions, are given in
italics.

\begin{figure}

\centering{

\pandocbounded{\includegraphics[keepaspectratio]{images/GlosBartonCemeteryExisting01a.png}}

}

\caption{\label{fig-03}Probability distributions of dates from
Gloucester: Barton cemetery. Each distribution represents the relative
probability that an event occurs at a particular time. For each of the
dates two distributions have been plotted: one in outline, which is the
result of simple radiocarbon calibration, and a solid one, based on the
chronological model used. The other component section of this model are
shown in detail in Figure~\ref{fig-04}. The large square brackets down
the left-hand side along with the OxCal keywords define the overall
model. exactly.}

\end{figure}%

\begin{figure}

\centering{

\pandocbounded{\includegraphics[keepaspectratio]{images/GlosBartonCemeteryExisting01B.png}}

}

\caption{\label{fig-04}Probability distributions of dates from
Gloucester: Barton cemetery. The format is identical to that of
Figure~\ref{fig-03} apart from the result followed by `?' that has been
excluded from the model. The large square brackets down the left-hand
side along with the OxCal keywords define the overall model.}

\end{figure}%

\begin{tcolorbox}[enhanced jigsaw, arc=.35mm, rightrule=.15mm, bottomrule=.15mm, toptitle=1mm, breakable, colframe=quarto-callout-note-color-frame, opacityback=0, toprule=.15mm, left=2mm, coltitle=black, colbacktitle=quarto-callout-note-color!10!white, bottomtitle=1mm, leftrule=.75mm, opacitybacktitle=0.6, title=\textcolor{quarto-callout-note-color}{\faInfo}\hspace{0.5em}{Expand to get the OxCal code for the model shown in Figures 3 and 4}, titlerule=0mm, colback=white]

\begin{Shaded}
\begin{Highlighting}[]
\NormalTok{ // Model loosely based on straigraphy in Holbrook et al. 2024}
\NormalTok{ // BRAMS{-}4175 excluded for reason outlined in text}
\NormalTok{ // 15 simulated dates between AD 250 and AD 350 }
\NormalTok{ Options()}
\NormalTok{ \{}
\NormalTok{  Resolution=1;}
\NormalTok{ \};}
\NormalTok{ Plot()}
\NormalTok{ \{}
\NormalTok{  Sequence("Gloucester: Barton cemtery")}
\NormalTok{  \{}
\NormalTok{   Boundary("StartBartonCemteryActivity");}
\NormalTok{   Phase("Barton Cemtery")}
\NormalTok{   \{}
\NormalTok{    Sequence("Barton Cemtery")}
\NormalTok{    \{}
\NormalTok{     Phase("Period 1")}
\NormalTok{     \{}
\NormalTok{      R\_Date("SUERC{-}58717", 1874, 28);}
\NormalTok{      After("Period 1 Media studies coins")}
\NormalTok{      \{}
\NormalTok{       U("Ditch 7965", 85, 96, 1);}
\NormalTok{      \};}
\NormalTok{     \};}
\NormalTok{     Phase("Period 2")}
\NormalTok{     \{}
\NormalTok{      Sequence("Cemetery")}
\NormalTok{      \{}
\NormalTok{       Boundary("StartBartonCemetery");}
\NormalTok{       Phase("Cemetery")}
\NormalTok{       \{}
\NormalTok{        Phase("Media Studies")}
\NormalTok{        \{}
\NormalTok{         First("FirstMediaStudiesburial");}
\NormalTok{         R\_Date("SUERC{-}58711", 1748, 29);}
\NormalTok{         R\_Date("SUERC{-}58712", 1855, 31);}
\NormalTok{         R\_Date("SUERC{-}58713", 1759, 29);}
\NormalTok{         R\_Date("SUERC{-}59634", 1664, 29);}
\NormalTok{         R\_Date("SUERC{-}59806", 1713, 27);}
\NormalTok{         After("Period 2 Media Studies Coins")}
\NormalTok{         \{}
\NormalTok{          U("Sk6133", 260, 290, 1);}
\NormalTok{          U("Sk6133", 260, 290, 1);}
\NormalTok{          U("Sk6137", 270, 290, 1);}
\NormalTok{          U("Sk6175", 313.0, 315.0, 0.1);}
\NormalTok{          U("Sk6326", 270, 290, 1);}
\NormalTok{          U("Sk6550", 260, 290, 1);}
\NormalTok{          U("Sk6601", 270, 290, 1);}
\NormalTok{          U("Sk6604", 270, 290, 1);}
\NormalTok{          U("Sk6903", 260, 290, 1);}
\NormalTok{          U("Sk7082", 286, 293, 1);}
\NormalTok{          U("Sk6685", 260, 290, 1);}
\NormalTok{         \};}
\NormalTok{         Span("MediaStudiesburial");}
\NormalTok{         Last("LastMediaStudiesburial");}
\NormalTok{        \};}
\NormalTok{        Phase("Block M")}
\NormalTok{        \{}
\NormalTok{         First("FirstBurialBlockM");}
\NormalTok{         R\_Date("BRAMS{-}4172", 1751, 25);}
\NormalTok{         R\_Date("BRAMS{-}4173", 1778, 25);}
\NormalTok{         R\_Date("BRAMS{-}4174", 1739, 25);}
\NormalTok{         R\_Date("BRAMS{-}4175", 1967, 27)}
\NormalTok{         \{}
\NormalTok{          Outlier();}
\NormalTok{         \};}
\NormalTok{         R\_Simulate("AD 285", 285, 20);}
\NormalTok{         R\_Simulate("AD 349", 349, 20);}
\NormalTok{         R\_Simulate("AD 269", 269, 20);}
\NormalTok{         R\_Simulate("AD 256", 256, 20);}
\NormalTok{         R\_Simulate("AD 288", 288, 20);}
\NormalTok{         R\_Simulate("AD 282", 282, 20);}
\NormalTok{         R\_Simulate("AD 298", 298, 20);}
\NormalTok{         R\_Simulate("AD 260", 260, 20);}
\NormalTok{         R\_Simulate("AD 307", 307, 20);}
\NormalTok{         R\_Simulate("AD 271", 271, 20);}
\NormalTok{         R\_Simulate("AD 287", 287, 20);}
\NormalTok{         R\_Simulate("AD 274", 274, 20);}
\NormalTok{         R\_Simulate("AD 324", 324, 20);}
\NormalTok{         R\_Simulate("AD 272", 272, 20);}
\NormalTok{         R\_Simulate("AD 284", 284, 20);}
\NormalTok{         After("Block M Coins")}
\NormalTok{         \{}
\NormalTok{          U("Sk1119", 348, 350, 1);}
\NormalTok{          U("Sk1873", 350, 353, 1);}
\NormalTok{          U("Sk1519", 69, 79, 1);}
\NormalTok{          U("Sk2017", 187.0, 189.0, 0.1);}
\NormalTok{          U("Sk1831", 350, 353, 1);}
\NormalTok{          U("Sk1474", 364, 378, 1);}
\NormalTok{         \};}
\NormalTok{         Span("BlockMburial");}
\NormalTok{         Last("LastBurialBlockM");}
\NormalTok{        \};}
\NormalTok{       \};}
\NormalTok{       Span("BartonCemetery");}
\NormalTok{       Boundary("EndBartonCemetery");}
\NormalTok{      \};}
\NormalTok{      Phase("Well 7481")}
\NormalTok{      \{}
\NormalTok{       R\_Date("SUERC{-}58710", 1649, 31);}
\NormalTok{      \};}
\NormalTok{      Phase("Deposit 6569")}
\NormalTok{      \{}
\NormalTok{       R\_Date("SUERC{-}61221", 1927, 29);}
\NormalTok{      \};}
\NormalTok{      After("Period 2 Media studies coins")}
\NormalTok{      \{}
\NormalTok{       U("Layer 6250", 364, 378, 1);}
\NormalTok{      \};}
\NormalTok{      After("Period 2 Block M coins")}
\NormalTok{      \{}
\NormalTok{       U("Ditch E", 69, 79, 1);}
\NormalTok{       U("layer 1350", 81, 96, 1);}
\NormalTok{       U("layer 1358", 85.0, 87.0, 01);}
\NormalTok{       U("layer 1002", 343, 348, 1);}
\NormalTok{       U("layer 1352", 364, 348, 1);}
\NormalTok{      \};}
\NormalTok{     \};}
\NormalTok{     Phase("Period 3")}
\NormalTok{     \{}
\NormalTok{      After("Pit 6108")}
\NormalTok{      \{}
\NormalTok{       R\_Date("SUERC{-}61220", 1920, 29);}
\NormalTok{      \};}
\NormalTok{      After("Period 3 Media studies coins")}
\NormalTok{      \{}
\NormalTok{       U("Pit 7803", 323.0, 325.0, 0.1);}
\NormalTok{       U("Layer 7803", 364, 378, 1);}
\NormalTok{      \};}
\NormalTok{     \};}
\NormalTok{     Phase("Period 4")}
\NormalTok{     \{}
\NormalTok{      R\_Date("SUERC{-}61222", 1100, 27);}
\NormalTok{      After("Period 4 Media studies coins")}
\NormalTok{      \{}
\NormalTok{       U("Layer 6070", 388, 402, 1);}
\NormalTok{       U("Cleaning layer 6077", 267.0, 269.0, 0.1);}
\NormalTok{      \};}
\NormalTok{     \};}
\NormalTok{     Phase("Period 5")}
\NormalTok{     \{}
\NormalTok{      After("Period 5 Block M coins")}
\NormalTok{      \{}
\NormalTok{       U("Soil layer 1276", 330, 335, 1);}
\NormalTok{       U("Soil layer 1276", 270, 273, 1);}
\NormalTok{       U("Ditch 1705", 364, 378, 1);}
\NormalTok{       U("Ditch 1296", 318, 324, 1);}
\NormalTok{       U("Posthole 1114", 307, 318, 1);}
\NormalTok{      \};}
\NormalTok{     \};}
\NormalTok{     Phase("Period 6")}
\NormalTok{     \{}
\NormalTok{      After("Period 6 Block M coins")}
\NormalTok{      \{}
\NormalTok{       U("Garden soil 1326", 367, 375, 1);}
\NormalTok{       U("Garden soil 1358", 367, 378, 1);}
\NormalTok{       U("Garden soil 1358", 77.0, 79.0, 0.1);}
\NormalTok{       U("Garden soil 1358", 81, 96, 1);}
\NormalTok{       U("Garden soil 1358", 293, 296, 1);}
\NormalTok{       U("Garden soil 1358", 270, 290, 1);}
\NormalTok{       U("Garden soil 1358", 270, 291, 1);}
\NormalTok{       U("Garden soil 1358", 270, 290, 1);}
\NormalTok{       U("Garden soil 1358", 270, 290, 1);}
\NormalTok{       U("Garden soil 1358", 270, 290, 1);}
\NormalTok{       U("Garden soil 1358", 270, 290, 1);}
\NormalTok{       U("Garden soil 1358", 270, 290, 1);}
\NormalTok{       U("Garden soil 1358", 322, 233, 1);}
\NormalTok{       U("Garden soil 1358", 330, 335, 1);}
\NormalTok{       U("Garden soil 1358", 330, 335, 1);}
\NormalTok{       U("Garden soil 1358", 330, 335, 1);}
\NormalTok{       U("Garden soil 1358", 367, 375, 1);}
\NormalTok{      \};}
\NormalTok{     \};}
\NormalTok{    \};}
\NormalTok{    Phase("Unstratified coins")}
\NormalTok{    \{}
\NormalTok{     After("Block M coins")}
\NormalTok{     \{}
\NormalTok{      U("", 386, 293, 1);}
\NormalTok{      U("", 318, 324, 1);}
\NormalTok{     \};}
\NormalTok{     After("Media Studies coins")}
\NormalTok{     \{}
\NormalTok{      U("Cleaning layer 6150", 66.0, 68.0, 0.1);}
\NormalTok{      U("Cleaning layer 6077", 270, 290, 1);}
\NormalTok{      U("Cleaning layer 6077", 267, 269, 0.1);}
\NormalTok{      U("Unstrat", 268, 270, 1);}
\NormalTok{     \};}
\NormalTok{    \};}
\NormalTok{    Span("BartonOverall");}
\NormalTok{   \};}
\NormalTok{   Boundary("EndBartonCemteryActivity");}
\NormalTok{  \};}
\NormalTok{ \};}
\end{Highlighting}
\end{Shaded}

\end{tcolorbox}

\section{Simulations}\label{simulations}

\subsection{Introduction}\label{introduction-1}

We have constructed simulation models to determine the potential of
further radiocarbon determinations from Gloucester Barton cemetery for
improving the precision of estimates for the chronology of the cemetery.

The components of a simulation model are those of any model. First the
available informative prior beliefs are established, from the model of
existing dates (see Figure~\ref{fig-03} \& Figure~\ref{fig-04}) and from
the stratigraphic matrix of suitable samples. After this, radiocarbon
dates can be simulated from the pool of suitable datable material and
the appropriate prior information incorporated into the simulation
model. Errors on the measurements are estimated from those recently
obtained by the selected laboratory on similar material of similar age.
In this process the actual date of the site has to be fed into the
model, which is done on the basis of our existing understanding of the
site chronology. Multiple models can be run for different actual ages
and for different sampling strategies to see which approach might be
most effective. Some examples of the simulations created during the
process of identifying the number of samples required for this project
are shown in Figure 2.

\subsection{Methodology}\label{methodology}

Calendar dates for each model were obtained using random number
generation. This is a process of creating a sequence of numbers that
don't follow any predictable pattern. They are widely used in
simulations, cryptography and statistical modelling. We used the
sample() function (2024)( to generate random integers (i.e.~the dates)
by sampling from a specified range. Below is the syntax:

\begin{Shaded}
\begin{Highlighting}[]
\FunctionTok{sample}\NormalTok{(}\DecValTok{250}\SpecialCharTok{:}\DecValTok{350}\NormalTok{, }\DecValTok{15}\NormalTok{, }\AttributeTok{replace =} \ConstantTok{FALSE}\NormalTok{)}
\end{Highlighting}
\end{Shaded}

\begin{verbatim}
 [1] 262 332 250 253 270 296 338 260 334 347 297 313 301 290 312
\end{verbatim}

Where: • x: Vector of numbers to sample from (eg AD 250--350);

• size: Number of values to return (15: the number of inhumations being
sample for isotope \& aDNA));

• replace: Whether sampling should be with replacement (FALSE).

We then used this code as part of a large piece of code (see below) to
generate a series of calendar dates for the R\_Simulate function in
OxCal (Bronk Ramsey, 2009) for simulations from 15 dates down to 2 dates
and incorporated them into a series of models. The models are based on
the assumption that the Block M part of the cemetery was in continuous
use for a period of time ((Buck et al., 1992)) {[}100 years{]}, between
AD 250 and AD 350. Time and resources, given this exercise required the
running of OxCal 140 models, were limited and we therefore were unable
to evaluate what potential differences in the duration of burial
activity and when it began and finished made to our date estimate.

\begin{Shaded}
\begin{Highlighting}[]
\FunctionTok{library}\NormalTok{(openxlsx)}
\FunctionTok{library}\NormalTok{(readxl)}
\FunctionTok{library}\NormalTok{(writexl)}
\FunctionTok{library}\NormalTok{(dplyr)}


\CommentTok{\# Create a new workbook}
\NormalTok{wb }\OtherTok{\textless{}{-}} \FunctionTok{createWorkbook}\NormalTok{()}

\CommentTok{\# Loop from 15 dates down to 2 dates}
\ControlFlowTok{for}\NormalTok{(n }\ControlFlowTok{in} \DecValTok{15}\SpecialCharTok{:}\DecValTok{2}\NormalTok{) \{}
  \CommentTok{\# Simulate n dates between AD 250 and AD 350}
\NormalTok{  dates }\OtherTok{\textless{}{-}} \FunctionTok{sample}\NormalTok{(}\DecValTok{250}\SpecialCharTok{:}\DecValTok{350}\NormalTok{, n, }\AttributeTok{replace =} \ConstantTok{FALSE}\NormalTok{)}
  
  \CommentTok{\# Debug: check what dates contains}
  \FunctionTok{print}\NormalTok{(}\FunctionTok{paste}\NormalTok{(}\StringTok{"n ="}\NormalTok{, n, }\StringTok{"dates ="}\NormalTok{, }\FunctionTok{paste}\NormalTok{(dates, }\AttributeTok{collapse =} \StringTok{", "}\NormalTok{)))}
  
  \CommentTok{\# Create a data frame}
\NormalTok{  df }\OtherTok{\textless{}{-}} \FunctionTok{data.frame}\NormalTok{(}\AttributeTok{Simulated\_Dates =}\NormalTok{ dates)}
  
  \CommentTok{\# Create sheet name}
\NormalTok{  sheet\_name }\OtherTok{\textless{}{-}} \FunctionTok{paste0}\NormalTok{(n, }\StringTok{"\_simulated\_dates"}\NormalTok{)}
  
  \CommentTok{\# Add worksheet and write data}
  \FunctionTok{addWorksheet}\NormalTok{(wb, sheet\_name)}
  \FunctionTok{writeData}\NormalTok{(wb, sheet\_name, df)}
\NormalTok{\}}

\CommentTok{\# Save the workbook}
\FunctionTok{saveWorkbook}\NormalTok{(wb, }\StringTok{"GlosBartonCemetery\_simulated\_dates.xlsx"}\NormalTok{, }\AttributeTok{overwrite =} \ConstantTok{TRUE}\NormalTok{)}
\end{Highlighting}
\end{Shaded}

Figure~\ref{fig-05} shows an extract of the full model, Block M, with
the 15 simulated dates for burials shown (R\_Simulate\ldots.). Each
model was run 10 times.

\begin{figure}

\centering{

\pandocbounded{\includegraphics[keepaspectratio]{images/GlosBartonCemeterySimulation01.png}}

}

\caption{\label{fig-05}Part of OxCal simulation: Block M with 15
simulated burial dates (R\_Simulate 2 simulated dates AD 250-350)}

\end{figure}%

\section{Analysis of simulation
outputs}\label{analysis-of-simulation-outputs}

\subsection{Extracting output from OxCal
models}\label{extracting-output-from-oxcal-models}

We extracted output from the OxCal models using the R code (see below).
Running each model ten times as part of one OxCal analysis results in a
considerable amount of data; hopefully the code makes this process less
time consuming.

\begin{Shaded}
\begin{Highlighting}[]
\FunctionTok{library}\NormalTok{(openxlsx)}
\FunctionTok{library}\NormalTok{(readxl)}
\FunctionTok{library}\NormalTok{(writexl)}
\FunctionTok{library}\NormalTok{(dplyr)}


\CommentTok{\# R code to extract @FirstBurialBlockM \& @LastBurialBlockM estimates from Barton Cemetery simulation file}
\CommentTok{\# This is an example for extracting the parameters @FirstBurialBlockM and @LastBurialBlockM from a single OxCal files (10 models) with n simulated dates, You need to repeat for each OxCal model}

\CommentTok{\# Read the file (suppress warning about incomplete final line)}
\NormalTok{file\_path }\OtherTok{\textless{}{-}} \StringTok{"GlosBartonCemeterySimulation06.txt"} \CommentTok{\#You will need to change the file name}
\NormalTok{data\_lines }\OtherTok{\textless{}{-}} \FunctionTok{readLines}\NormalTok{(file\_path, }\AttributeTok{warn =} \ConstantTok{FALSE}\NormalTok{)}

\CommentTok{\# Extract lines containing @FirstBurialBlockM}
\NormalTok{burial\_lines }\OtherTok{\textless{}{-}}\NormalTok{ data\_lines[}\FunctionTok{grep}\NormalTok{(}\StringTok{"@FirstBurialBlockM"}\NormalTok{, data\_lines)]}

\CommentTok{\# Display the found lines}
\FunctionTok{cat}\NormalTok{(}\StringTok{"Found"}\NormalTok{, }\FunctionTok{length}\NormalTok{(burial\_lines), }\StringTok{"@FirstBurialBlockM entries:}\SpecialCharTok{\textbackslash{}n}\StringTok{"}\NormalTok{)}
\FunctionTok{print}\NormalTok{(burial\_lines)}

\CommentTok{\# Extract numerical values from @FirstBurialBlockM lines}

\CommentTok{\# Initialize vectors for different columns based on the visible pattern}
\NormalTok{burial\_data }\OtherTok{\textless{}{-}} \FunctionTok{data.frame}\NormalTok{(}
  \AttributeTok{line\_number =} \FunctionTok{integer}\NormalTok{(),}
  \AttributeTok{estimate\_1 =} \FunctionTok{numeric}\NormalTok{(),}
  \AttributeTok{estimate\_2 =} \FunctionTok{numeric}\NormalTok{(),}
  \AttributeTok{estimate\_3 =} \FunctionTok{numeric}\NormalTok{(),}
  \AttributeTok{estimate\_4 =} \FunctionTok{numeric}\NormalTok{(),}
  \AttributeTok{stringsAsFactors =} \ConstantTok{FALSE}
\NormalTok{)}

\CommentTok{\# Process each @FirstBurialBlockM line}
\ControlFlowTok{for}\NormalTok{ (i }\ControlFlowTok{in} \FunctionTok{seq\_along}\NormalTok{(burial\_lines)) \{}
\NormalTok{  line }\OtherTok{\textless{}{-}}\NormalTok{ burial\_lines[i]}
  
  \CommentTok{\# Extract all numbers from the line}
\NormalTok{  numbers }\OtherTok{\textless{}{-}} \FunctionTok{regmatches}\NormalTok{(line, }\FunctionTok{gregexpr}\NormalTok{(}\StringTok{"{-}?}\SpecialCharTok{\textbackslash{}\textbackslash{}}\StringTok{d+}\SpecialCharTok{\textbackslash{}\textbackslash{}}\StringTok{.?}\SpecialCharTok{\textbackslash{}\textbackslash{}}\StringTok{d*"}\NormalTok{, line))[[}\DecValTok{1}\NormalTok{]]}
\NormalTok{  numbers }\OtherTok{\textless{}{-}} \FunctionTok{as.numeric}\NormalTok{(numbers)}
  
  \CommentTok{\# Create a row for the data frame (pad with NA if fewer than 4 values)}
\NormalTok{  row\_data }\OtherTok{\textless{}{-}} \FunctionTok{c}\NormalTok{(i, numbers, }\FunctionTok{rep}\NormalTok{(}\ConstantTok{NA}\NormalTok{, }\DecValTok{4} \SpecialCharTok{{-}} \FunctionTok{length}\NormalTok{(numbers)))[}\DecValTok{1}\SpecialCharTok{:}\DecValTok{5}\NormalTok{]}
\NormalTok{  burial\_data }\OtherTok{\textless{}{-}} \FunctionTok{rbind}\NormalTok{(burial\_data, row\_data)}
\NormalTok{\}}

\CommentTok{\# Set proper column names}
\FunctionTok{colnames}\NormalTok{(burial\_data) }\OtherTok{\textless{}{-}} \FunctionTok{c}\NormalTok{(}\StringTok{"Run"}\NormalTok{, }\StringTok{"68\%start"}\NormalTok{, }\StringTok{"68\%end"}\NormalTok{, }\StringTok{"95\%start"}\NormalTok{, }\StringTok{"95\%end"}\NormalTok{)}

\CommentTok{\# Display the structured data}
\FunctionTok{cat}\NormalTok{(}\StringTok{"}\SpecialCharTok{\textbackslash{}n}\StringTok{Structured burial estimates:}\SpecialCharTok{\textbackslash{}n}\StringTok{"}\NormalTok{)}
\FunctionTok{print}\NormalTok{(burial\_data)}


\NormalTok{burial\_data }\OtherTok{\textless{}{-}}\NormalTok{ burial\_data }\SpecialCharTok{\%\textgreater{}\%}
  \FunctionTok{mutate}\NormalTok{(}\AttributeTok{range\_68 =} \StringTok{\textasciigrave{}}\AttributeTok{68\%end}\StringTok{\textasciigrave{}} \SpecialCharTok{{-}} \StringTok{\textasciigrave{}}\AttributeTok{68\%start}\StringTok{\textasciigrave{}}\NormalTok{,}
         \AttributeTok{range\_95 =} \StringTok{\textasciigrave{}}\AttributeTok{95\%end}\StringTok{\textasciigrave{}} \SpecialCharTok{{-}} \StringTok{\textasciigrave{}}\AttributeTok{95\%start}\StringTok{\textasciigrave{}}\NormalTok{)}
\FunctionTok{print}\NormalTok{(burial\_data)}

\CommentTok{\# Save the data}
\FunctionTok{write\_xlsx}\NormalTok{(burial\_data, }\StringTok{"BartonBlockM06starts.xlsx"}\NormalTok{)}

\CommentTok{\# PART 2}
\CommentTok{\# Extract lines containing @LastBurialBlockM}
\NormalTok{burial\_lines2 }\OtherTok{\textless{}{-}}\NormalTok{ data\_lines[}\FunctionTok{grep}\NormalTok{(}\StringTok{"@LastBurialBlockM"}\NormalTok{, data\_lines)]}

\CommentTok{\# Display the found lines}
\FunctionTok{cat}\NormalTok{(}\StringTok{"Found"}\NormalTok{, }\FunctionTok{length}\NormalTok{(burial\_lines2), }\StringTok{"@LastBurialBlockM entries:}\SpecialCharTok{\textbackslash{}n}\StringTok{"}\NormalTok{)}
\FunctionTok{print}\NormalTok{(burial\_lines)}

\CommentTok{\# Extract numerical values from @FirstBurialBlockM lines}

\CommentTok{\# Initialize vectors for different columns based on the visible pattern}
\NormalTok{burial\_data2 }\OtherTok{\textless{}{-}} \FunctionTok{data.frame}\NormalTok{(}
  \AttributeTok{line\_number =} \FunctionTok{integer}\NormalTok{(),}
  \AttributeTok{estimate\_1 =} \FunctionTok{numeric}\NormalTok{(),}
  \AttributeTok{estimate\_2 =} \FunctionTok{numeric}\NormalTok{(),}
  \AttributeTok{estimate\_3 =} \FunctionTok{numeric}\NormalTok{(),}
  \AttributeTok{estimate\_4 =} \FunctionTok{numeric}\NormalTok{(),}
  \AttributeTok{stringsAsFactors =} \ConstantTok{FALSE}
\NormalTok{)}

\CommentTok{\# Process each @LastBurialBlockM line}
\ControlFlowTok{for}\NormalTok{ (i }\ControlFlowTok{in} \FunctionTok{seq\_along}\NormalTok{(burial\_lines2)) \{}
\NormalTok{  line }\OtherTok{\textless{}{-}}\NormalTok{ burial\_lines2[i]}
  
  \CommentTok{\# Extract all numbers from the line}
\NormalTok{  numbers }\OtherTok{\textless{}{-}} \FunctionTok{regmatches}\NormalTok{(line, }\FunctionTok{gregexpr}\NormalTok{(}\StringTok{"{-}?}\SpecialCharTok{\textbackslash{}\textbackslash{}}\StringTok{d+}\SpecialCharTok{\textbackslash{}\textbackslash{}}\StringTok{.?}\SpecialCharTok{\textbackslash{}\textbackslash{}}\StringTok{d*"}\NormalTok{, line))[[}\DecValTok{1}\NormalTok{]]}
\NormalTok{  numbers }\OtherTok{\textless{}{-}} \FunctionTok{as.numeric}\NormalTok{(numbers)}
  
  \CommentTok{\# Create a row for the data frame (pad with NA if fewer than 4 values)}
\NormalTok{  row\_data }\OtherTok{\textless{}{-}} \FunctionTok{c}\NormalTok{(i, numbers, }\FunctionTok{rep}\NormalTok{(}\ConstantTok{NA}\NormalTok{, }\DecValTok{4} \SpecialCharTok{{-}} \FunctionTok{length}\NormalTok{(numbers)))[}\DecValTok{1}\SpecialCharTok{:}\DecValTok{5}\NormalTok{]}
\NormalTok{  burial\_data2 }\OtherTok{\textless{}{-}} \FunctionTok{rbind}\NormalTok{(burial\_data2, row\_data)}
\NormalTok{\}}

\CommentTok{\# Set proper column names}
\FunctionTok{colnames}\NormalTok{(burial\_data2) }\OtherTok{\textless{}{-}} \FunctionTok{c}\NormalTok{(}\StringTok{"Run"}\NormalTok{, }\StringTok{"68\%start"}\NormalTok{, }\StringTok{"68\%end"}\NormalTok{, }\StringTok{"95\%start"}\NormalTok{, }\StringTok{"95\%end"}\NormalTok{)}

\CommentTok{\# Display the structured data}
\FunctionTok{cat}\NormalTok{(}\StringTok{"}\SpecialCharTok{\textbackslash{}n}\StringTok{Structured burial estimates:}\SpecialCharTok{\textbackslash{}n}\StringTok{"}\NormalTok{)}
\FunctionTok{print}\NormalTok{(burial\_data2)}


\NormalTok{burial\_data2 }\OtherTok{\textless{}{-}}\NormalTok{ burial\_data2 }\SpecialCharTok{\%\textgreater{}\%}
  \FunctionTok{mutate}\NormalTok{(}\AttributeTok{range\_68 =} \StringTok{\textasciigrave{}}\AttributeTok{68\%end}\StringTok{\textasciigrave{}} \SpecialCharTok{{-}} \StringTok{\textasciigrave{}}\AttributeTok{68\%start}\StringTok{\textasciigrave{}}\NormalTok{,}
         \AttributeTok{range\_95 =} \StringTok{\textasciigrave{}}\AttributeTok{95\%end}\StringTok{\textasciigrave{}} \SpecialCharTok{{-}} \StringTok{\textasciigrave{}}\AttributeTok{95\%start}\StringTok{\textasciigrave{}}\NormalTok{)}
\FunctionTok{print}\NormalTok{(burial\_data2)}

\CommentTok{\# Save the data}
\FunctionTok{write\_xlsx}\NormalTok{(burial\_data2, }\StringTok{"BartonBlockM06ends.xlsx"}\NormalTok{)}
\end{Highlighting}
\end{Shaded}

\subsection{Visualisation of output from OxCal
models}\label{visualisation-of-output-from-oxcal-models}

Figure~\ref{fig-06} and Figure~\ref{fig-07} show a summary of the date
estimate for the parameter \emph{FirstBlockMBurial}. It is clear,
particularly, on Figure~\ref{fig-06} that the submission of more
radiocarbon samples does not increase precision. In fact the submission
of further samples could increase the bandwith of the date estimate (eg,
13 \& 14 samples).

\begin{figure}

\centering{

\pandocbounded{\includegraphics[keepaspectratio]{images/BlockMFirst.png}}

}

\caption{\label{fig-06}Summary of 140 simulation models; 10 models for
15\ldots.2 simulated dates AD 250-350, showing precision vs sample size
for \emph{FirstBlockMBurial}}

\end{figure}%

\begin{Shaded}
\begin{Highlighting}[]
\CommentTok{\# Visualizing differences across multiple sample sizes (15, 14, 13...2 samples)}

\FunctionTok{library}\NormalTok{(ggplot2)}
\FunctionTok{library}\NormalTok{(dplyr)}
\FunctionTok{library}\NormalTok{(tidyr)}
\FunctionTok{library}\NormalTok{(readxl)}
\FunctionTok{library}\NormalTok{(gridExtra)}
\FunctionTok{library}\NormalTok{(viridis)}

\CommentTok{\# Read the data}
\NormalTok{data }\OtherTok{\textless{}{-}} \FunctionTok{read\_excel}\NormalTok{(}\StringTok{"Query.xlsx"}\NormalTok{, }\AttributeTok{sheet =} \StringTok{"BlockMStarts"}\NormalTok{) }\CommentTok{\# change data to name of site name or something else}

\CommentTok{\# Extract sample size from Parameter column (assumes format like "15Start", "14Start", etc.)}
\NormalTok{data}\SpecialCharTok{$}\NormalTok{Sample\_Size }\OtherTok{\textless{}{-}} \FunctionTok{as.numeric}\NormalTok{(}\FunctionTok{gsub}\NormalTok{(}\StringTok{"Start"}\NormalTok{, }\StringTok{""}\NormalTok{, data}\SpecialCharTok{$}\NormalTok{Parameter))}
\NormalTok{data}\SpecialCharTok{$}\NormalTok{Sample\_Size\_Label }\OtherTok{\textless{}{-}} \FunctionTok{paste0}\NormalTok{(data}\SpecialCharTok{$}\NormalTok{Sample\_Size, }\StringTok{" samples"}\NormalTok{)}
\NormalTok{data}\SpecialCharTok{$}\NormalTok{Model\_Run }\OtherTok{\textless{}{-}} \FunctionTok{ave}\NormalTok{(}\FunctionTok{seq\_along}\NormalTok{(data}\SpecialCharTok{$}\NormalTok{Parameter), data}\SpecialCharTok{$}\NormalTok{Parameter, }\AttributeTok{FUN =}\NormalTok{ seq\_along)}

\CommentTok{\# Create a duration column (End {-} Start)}
\NormalTok{data}\SpecialCharTok{$}\NormalTok{Duration }\OtherTok{\textless{}{-}}\NormalTok{ data}\SpecialCharTok{$}\NormalTok{End }\SpecialCharTok{{-}}\NormalTok{ data}\SpecialCharTok{$}\NormalTok{Start}

\CommentTok{\# TREND ANALYSIS {-} Shows how precision changes with sample size}
\NormalTok{summary\_stats }\OtherTok{\textless{}{-}}\NormalTok{ data }\SpecialCharTok{\%\textgreater{}\%}
  \FunctionTok{group\_by}\NormalTok{(Sample\_Size) }\SpecialCharTok{\%\textgreater{}\%}
  \FunctionTok{summarise}\NormalTok{(}
    \AttributeTok{Mean\_Start =} \FunctionTok{mean}\NormalTok{(Start),}
    \AttributeTok{SD\_Start =} \FunctionTok{sd}\NormalTok{(Start),}
    \AttributeTok{Mean\_End =} \FunctionTok{mean}\NormalTok{(End),}
    \AttributeTok{SD\_End =} \FunctionTok{sd}\NormalTok{(End),}
    \AttributeTok{Mean\_Duration =} \FunctionTok{mean}\NormalTok{(Duration),}
    \AttributeTok{SD\_Duration =} \FunctionTok{sd}\NormalTok{(Duration),}
    \AttributeTok{CV\_Start =} \FunctionTok{sd}\NormalTok{(Start)}\SpecialCharTok{/}\FunctionTok{mean}\NormalTok{(Start) }\SpecialCharTok{*} \DecValTok{100}\NormalTok{,  }\CommentTok{\# Coefficient of variation}
    \AttributeTok{.groups =} \StringTok{\textquotesingle{}drop\textquotesingle{}}
\NormalTok{  )}

\NormalTok{BlockMFirst }\OtherTok{\textless{}{-}} \FunctionTok{ggplot}\NormalTok{(summary\_stats, }\FunctionTok{aes}\NormalTok{(}\AttributeTok{x =}\NormalTok{ Sample\_Size)) }\SpecialCharTok{+}
  \FunctionTok{geom\_line}\NormalTok{(}\FunctionTok{aes}\NormalTok{(}\AttributeTok{y =}\NormalTok{ Mean\_Start, }\AttributeTok{color =} \StringTok{"Start Date"}\NormalTok{), }\AttributeTok{size =} \FloatTok{1.2}\NormalTok{, }\AttributeTok{alpha =} \FloatTok{0.8}\NormalTok{) }\SpecialCharTok{+}
  \FunctionTok{geom\_point}\NormalTok{(}\FunctionTok{aes}\NormalTok{(}\AttributeTok{y =}\NormalTok{ Mean\_Start, }\AttributeTok{color =} \StringTok{"Start Date"}\NormalTok{), }\AttributeTok{size =} \DecValTok{3}\NormalTok{) }\SpecialCharTok{+}
  \FunctionTok{geom\_ribbon}\NormalTok{(}\FunctionTok{aes}\NormalTok{(}\AttributeTok{ymin =}\NormalTok{ Mean\_Start }\SpecialCharTok{{-}}\NormalTok{ SD\_Start, }\AttributeTok{ymax =}\NormalTok{ Mean\_Start }\SpecialCharTok{+}\NormalTok{ SD\_Start, }
                  \AttributeTok{fill =} \StringTok{"Start Date"}\NormalTok{), }\AttributeTok{alpha =} \FloatTok{0.3}\NormalTok{) }\SpecialCharTok{+}
  \FunctionTok{geom\_line}\NormalTok{(}\FunctionTok{aes}\NormalTok{(}\AttributeTok{y =}\NormalTok{ Mean\_End, }\AttributeTok{color =} \StringTok{"End Date"}\NormalTok{), }\AttributeTok{size =} \FloatTok{1.2}\NormalTok{, }\AttributeTok{alpha =} \FloatTok{0.8}\NormalTok{) }\SpecialCharTok{+}
  \FunctionTok{geom\_point}\NormalTok{(}\FunctionTok{aes}\NormalTok{(}\AttributeTok{y =}\NormalTok{ Mean\_End, }\AttributeTok{color =} \StringTok{"End Date"}\NormalTok{), }\AttributeTok{size =} \DecValTok{3}\NormalTok{) }\SpecialCharTok{+}
  \FunctionTok{geom\_ribbon}\NormalTok{(}\FunctionTok{aes}\NormalTok{(}\AttributeTok{ymin =}\NormalTok{ Mean\_End }\SpecialCharTok{{-}}\NormalTok{ SD\_End, }\AttributeTok{ymax =}\NormalTok{ Mean\_End }\SpecialCharTok{+}\NormalTok{ SD\_End, }
                  \AttributeTok{fill =} \StringTok{"End Date"}\NormalTok{), }\AttributeTok{alpha =} \FloatTok{0.3}\NormalTok{) }\SpecialCharTok{+}
  \FunctionTok{scale\_color\_manual}\NormalTok{(}\AttributeTok{values =} \FunctionTok{c}\NormalTok{(}\StringTok{"Start Date"} \OtherTok{=} \StringTok{"\#2E86AB"}\NormalTok{, }\StringTok{"End Date"} \OtherTok{=} \StringTok{"\#2E86AB"}\NormalTok{)) }\SpecialCharTok{+}
  \FunctionTok{scale\_fill\_manual}\NormalTok{(}\AttributeTok{values =} \FunctionTok{c}\NormalTok{(}\StringTok{"Start Date"} \OtherTok{=} \StringTok{"\#2E86AB"}\NormalTok{, }\StringTok{"End Date"} \OtherTok{=} \StringTok{"\#2E86AB"}\NormalTok{)) }\SpecialCharTok{+}
  \FunctionTok{scale\_x\_continuous}\NormalTok{(}\AttributeTok{breaks =} \FunctionTok{sort}\NormalTok{(}\FunctionTok{unique}\NormalTok{(data}\SpecialCharTok{$}\NormalTok{Sample\_Size))) }\SpecialCharTok{+}
  \FunctionTok{labs}\NormalTok{(}\AttributeTok{title =} \StringTok{"FirstBlockMBurial vs precision vs sample size"}\NormalTok{,}
       \AttributeTok{subtitle =} \StringTok{"Mean estimates with ±1 SD ribbons"}\NormalTok{,}
       \AttributeTok{x =} \StringTok{"Number of Samples"}\NormalTok{,}
       \AttributeTok{y =} \StringTok{"Posterior density estimate (cal AD)"}\NormalTok{,}
       \AttributeTok{color =} \StringTok{"Boundary"}\NormalTok{,}
       \AttributeTok{fill =} \StringTok{"Boundary"}\NormalTok{) }\SpecialCharTok{+}
  \FunctionTok{theme\_classic}\NormalTok{() }\SpecialCharTok{+}
  \FunctionTok{theme}\NormalTok{(}\AttributeTok{legend.position =} \StringTok{"none"}\NormalTok{)}

\FunctionTok{ggsave}\NormalTok{(}\StringTok{"BlockMFirst.png"}\NormalTok{, BlockMFirst, }
       \AttributeTok{width =} \DecValTok{8}\NormalTok{, }\AttributeTok{height =} \DecValTok{6}\NormalTok{, }\AttributeTok{dpi =} \DecValTok{300}\NormalTok{)}
\end{Highlighting}
\end{Shaded}

\begin{figure}

\centering{

\pandocbounded{\includegraphics[keepaspectratio]{images/BlockMFirstBurial.png}}

}

\caption{\label{fig-07}Summary of 140 simulation models; 10 models for
15\ldots.2 simulated dates AD 250-350, showing bandwidth for
\emph{FirstBlockMBurial}}

\end{figure}%

\begin{Shaded}
\begin{Highlighting}[]
\CommentTok{\# Visualizing differences across multiple sample sizes (15, 14, 13...2 samples)}

\FunctionTok{library}\NormalTok{(ggplot2)}
\FunctionTok{library}\NormalTok{(dplyr)}
\FunctionTok{library}\NormalTok{(tidyr)}
\FunctionTok{library}\NormalTok{(readxl)}
\FunctionTok{library}\NormalTok{(gridExtra)}
\FunctionTok{library}\NormalTok{(viridis)}

\CommentTok{\# Read the data}
\NormalTok{data }\OtherTok{\textless{}{-}} \FunctionTok{read\_excel}\NormalTok{(}\StringTok{"Query.xlsx"}\NormalTok{, }\AttributeTok{sheet =} \StringTok{"BlockMStarts"}\NormalTok{) }\CommentTok{\# change data to name of site name or something else}

\CommentTok{\# Extract sample size from Parameter column (assumes format like "15Start", "14Start", etc.)}
\NormalTok{data}\SpecialCharTok{$}\NormalTok{Sample\_Size }\OtherTok{\textless{}{-}} \FunctionTok{as.numeric}\NormalTok{(}\FunctionTok{gsub}\NormalTok{(}\StringTok{"Start"}\NormalTok{, }\StringTok{""}\NormalTok{, data}\SpecialCharTok{$}\NormalTok{Parameter))}
\NormalTok{data}\SpecialCharTok{$}\NormalTok{Sample\_Size\_Label }\OtherTok{\textless{}{-}} \FunctionTok{paste0}\NormalTok{(data}\SpecialCharTok{$}\NormalTok{Sample\_Size, }\StringTok{" samples"}\NormalTok{)}
\NormalTok{data}\SpecialCharTok{$}\NormalTok{Model\_Run }\OtherTok{\textless{}{-}} \FunctionTok{ave}\NormalTok{(}\FunctionTok{seq\_along}\NormalTok{(data}\SpecialCharTok{$}\NormalTok{Parameter), data}\SpecialCharTok{$}\NormalTok{Parameter, }\AttributeTok{FUN =}\NormalTok{ seq\_along)}

\CommentTok{\# Create a duration column (End {-} Start)}
\NormalTok{data}\SpecialCharTok{$}\NormalTok{Duration }\OtherTok{\textless{}{-}}\NormalTok{ data}\SpecialCharTok{$}\NormalTok{End }\SpecialCharTok{{-}}\NormalTok{ data}\SpecialCharTok{$}\NormalTok{Start}

\CommentTok{\# TIMELINE VISUALIZATION {-} Shows date ranges across all sample sizes}
\NormalTok{data}\SpecialCharTok{$}\NormalTok{Sample\_Group }\OtherTok{\textless{}{-}} \FunctionTok{factor}\NormalTok{(data}\SpecialCharTok{$}\NormalTok{Sample\_Size)}
\NormalTok{data}\SpecialCharTok{$}\NormalTok{y\_position }\OtherTok{\textless{}{-}} \FunctionTok{as.numeric}\NormalTok{(}\FunctionTok{factor}\NormalTok{(}\FunctionTok{paste0}\NormalTok{(data}\SpecialCharTok{$}\NormalTok{Sample\_Size, }\StringTok{"\_"}\NormalTok{, data}\SpecialCharTok{$}\NormalTok{Model\_Run)))}

\NormalTok{BlockMFirstBurial }\OtherTok{\textless{}{-}} \FunctionTok{ggplot}\NormalTok{(data, }\FunctionTok{aes}\NormalTok{(}\AttributeTok{y =} \FunctionTok{reorder}\NormalTok{(}\FunctionTok{paste0}\NormalTok{(data}\SpecialCharTok{$}\NormalTok{Sample\_Size, }\StringTok{" samples {-} Run "}\NormalTok{, data}\SpecialCharTok{$}\NormalTok{Model\_Run), }
                                   \SpecialCharTok{{-}}\NormalTok{y\_position))) }\SpecialCharTok{+}
  \FunctionTok{geom\_segment}\NormalTok{(}\FunctionTok{aes}\NormalTok{(}\AttributeTok{x =}\NormalTok{ Start, }\AttributeTok{xend =}\NormalTok{ End, }\AttributeTok{color =} \FunctionTok{factor}\NormalTok{(Sample\_Size)), }
               \AttributeTok{size =} \DecValTok{2}\NormalTok{, }\AttributeTok{alpha =} \FloatTok{0.7}\NormalTok{) }\SpecialCharTok{+}
  \FunctionTok{geom\_point}\NormalTok{(}\FunctionTok{aes}\NormalTok{(}\AttributeTok{x =}\NormalTok{ Start, }\AttributeTok{color =} \FunctionTok{factor}\NormalTok{(Sample\_Size)), }\AttributeTok{size =} \FloatTok{1.5}\NormalTok{) }\SpecialCharTok{+}
  \FunctionTok{geom\_point}\NormalTok{(}\FunctionTok{aes}\NormalTok{(}\AttributeTok{x =}\NormalTok{ End, }\AttributeTok{color =} \FunctionTok{factor}\NormalTok{(Sample\_Size)), }\AttributeTok{size =} \FloatTok{1.5}\NormalTok{) }\SpecialCharTok{+}
  \FunctionTok{scale\_color\_viridis\_d}\NormalTok{(}\AttributeTok{option =} \StringTok{"plasma"}\NormalTok{, }\AttributeTok{direction =} \SpecialCharTok{{-}}\DecValTok{1}\NormalTok{, }\AttributeTok{name =} \StringTok{"Sample Size"}\NormalTok{) }\SpecialCharTok{+}
  \FunctionTok{labs}\NormalTok{(}\AttributeTok{title =} \StringTok{"Barton Cemetery: FirstBurialBlockM by addional 14C measurements"}\NormalTok{,}
       \AttributeTok{subtitle =} \StringTok{"Bandwidth for each simulation"}\NormalTok{,}
       \AttributeTok{x =} \StringTok{"Date (cal AD)"}\NormalTok{,}
       \AttributeTok{y =} \StringTok{"Model Run"}\NormalTok{) }\SpecialCharTok{+}
  \FunctionTok{theme\_classic}\NormalTok{() }\SpecialCharTok{+}
  \FunctionTok{theme}\NormalTok{(}\AttributeTok{axis.text.y =} \FunctionTok{element\_text}\NormalTok{(}\AttributeTok{size =} \DecValTok{6}\NormalTok{),}
        \AttributeTok{legend.position =} \StringTok{"right"}\NormalTok{) }\SpecialCharTok{+}
  \FunctionTok{guides}\NormalTok{(}\AttributeTok{color =} \FunctionTok{guide\_legend}\NormalTok{(}\AttributeTok{override.aes =} \FunctionTok{list}\NormalTok{(}\AttributeTok{size =} \DecValTok{3}\NormalTok{)))}

\FunctionTok{ggsave}\NormalTok{(}\StringTok{"BlockMFirstBurial.png"}\NormalTok{, BlockMFirstBurial, }
       \AttributeTok{width =} \DecValTok{8}\NormalTok{, }\AttributeTok{height =} \DecValTok{6}\NormalTok{, }\AttributeTok{dpi =} \DecValTok{300}\NormalTok{)}
\end{Highlighting}
\end{Shaded}

Figure~\ref{fig-08} and Figure~\ref{fig-09} show a summary of the date
estimate for the parameter \emph{LastBlockMBurial}. It is clear,
particularly, on Figure~\ref{fig-09} that the submission of an 8
additional, or more radiocarbon samples does increase precision, note
how the two lines gets closer together. The increase in precision is
though not signficant.

\begin{figure}

\centering{

\pandocbounded{\includegraphics[keepaspectratio]{images/BlockMLast.png}}

}

\caption{\label{fig-08}Summary of 140 simulation models; 10 models for
15\ldots.2 simulated dates AD 250-350, showing precision vs sample size
for \emph{LastBlockMBurial}}

\end{figure}%

\begin{figure}

\centering{

\pandocbounded{\includegraphics[keepaspectratio]{images/BlockMLastBurial.png}}

}

\caption{\label{fig-09}Summary of 140 simulation models; 10 models for
15\ldots.2 simulated dates AD 250-350, showing bandwidth for
\emph{LastBlockMBurial}}

\end{figure}%

\section{Conclusion}\label{conclusion}

The submission of more samples from the Block M cemetery is no
recommended, unless specific burials are identified as particularly
waranting scientific dating, ie they are spatially isolated from the
main burial group, have unsual pathologies or isotopic signitures.

The analysis outlined above has been undertaken without access to the
full site records, ie matricies, startigraphic narrative and thus the
conlusions may need to be revised should this infomtayion become
avaialble int he future.

\section*{References}\label{references}
\addcontentsline{toc}{section}{References}

\phantomsection\label{refs}
\begin{CSLReferences}{1}{0}
\vspace{1em}

\bibitem[\citeproctext]{ref-bronkramsey2009}
Bronk Ramsey, C. (2009). Bayesian Analysis of Radiocarbon Dates.
\emph{Radiocarbon}, \emph{51}(1), 337--360.
\url{https://doi.org/10.1017/s0033822200033865}

\bibitem[\citeproctext]{ref-buck1992}
Buck, C. E., Litton, C. D., \& Smith, A. F. M. (1992). Calibration of
radiocarbon results pertaining to related archaeological events.
\emph{Journal of Archaeological Science}, \emph{19}(5), 497--512.
\url{https://doi.org/10.1016/0305-4403(92)90025-x}

\bibitem[\citeproctext]{ref-dunbar2016}
Dunbar, E., Cook, G. T., Naysmith, P., Tripney, B. G., \& Xu, S. (2016).
AMS {\textsuperscript{14}}C Dating at the Scottish Universities
Environmental Research Centre (SUERC) Radiocarbon Dating Laboratory
{\textendash} Corrigendum. \emph{Radiocarbon}, \emph{58}(1), 233--233.
\url{https://doi.org/10.1017/rdc.2016.14}

\bibitem[\citeproctext]{ref-knowles_et_al_2019}
Knowles, T. D. J., Monaghan, P. S., \& Evershed, R. P. (2019).
Radiocarbon Sample Preparation Procedures and the First Status Report
from the Bristol Radiocarbon AMS (BRAMS) Facility. \emph{Radiocarbon},
\emph{61}(5), 1541--1550. \url{https://doi.org/10.1017/rdc.2019.28}

\bibitem[\citeproctext]{ref-lanting_vanderplict_1988}
Lanting, J. N., \& Plicht, J. V. D. (1998). Reservoir effects and
apparent \(^{14}C-Ages\). \emph{The Journal of Irish Archaeology},
\emph{9}, 151--165. Retrieved from
\url{http://www.jstor.org/stable/30001698}

\bibitem[\citeproctext]{ref-base}
R Core Team. (2024). R: A language and environment for statistical
computing. Retrieved from \url{https://www.R-project.org/}

\bibitem[\citeproctext]{ref-reimer2020}
Reimer, P. J., Austin, W. E. N., Bard, E., Bayliss, A., Blackwell, P.
G., Bronk Ramsey, C., et al. (2020). The IntCal20 Northern Hemisphere
Radiocarbon Age Calibration Curve (0{\textendash}55 cal kBP).
\emph{Radiocarbon}, \emph{62}(4), 725--757.
\url{https://doi.org/10.1017/RDC.2020.41}

\bibitem[\citeproctext]{ref-StuiverPolach1977}
Stuiver, M., \& Polach, H. A. (1977). Discussion reporting of 14C data.
\emph{Radiocarbon}, \emph{19}(3), 355--363.
\url{https://doi.org/10.1017/S0033822200003672}

\bibitem[\citeproctext]{ref-synal2007}
Synal, H.-A., Stocker, M., \& Suter, M. (2007). MICADAS: A new compact
radiocarbon AMS system. \emph{Nuclear Instruments and Methods in Physics
Research Section B: Beam Interactions with Materials and Atoms},
\emph{259}(1), 7--13.
https://doi.org/\url{https://doi.org/10.1016/j.nimb.2007.01.138}

\end{CSLReferences}

::::




\end{document}
